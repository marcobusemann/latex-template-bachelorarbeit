\mbchapter{Frameworks}
\label{frameworks}

In diesem Kapitel werden die verwendeten Frameworks und ihre Versionen beschrieben. Die Frameworks bauen technisch aufeinander auf und können somit auf drei Ebenen angeordnet werden. Auf der niedrigsten Ebene befindet sich Cordova, als Schnittstelle zum Betriebssystem. Darüber ist AngularJS angesiedelt und gibt die Struktur der Anwendung vor. Auf der obersten Ebene befindet sich letztendlich Ionic. Ionic stellt die \gls{ui} Komponenten bereit, mit denen der Anwender interagiert. 

\section{Cordova}

Die Plattform Apache Cordova ermöglicht es, die mit Webtechnologien wie \gls{html}, JavaScript und \gls{css} erstellten Anwendung als App zu verpacken. Sie bildet dabei die Brücke zwischen dem Betriebssystem und der mit Webtechnologien entwickelten Anwendung. Die Anwendung wird in einer Komponente mit dem Namen \gls{WebView} angezeigt. Das \gls{WebView} wird vom Betriebssystem bereitgestellt und beeinflusst grundlegend die Geschwindigkeit der Anwendung. Über eine von Cordova bereitgestellte \glspl{api} haben Anwendungen die Möglichkeit, auf Funktionen des Betriebssystems sowie sonstige Programmierschnittstellen zuzugreifen \cite{cordova}.
Cordova unterstützt alle gängigen Handy- und Tablet-Betriebssysteme wie Android, iOS und Windows Phone. Zum Zeitpunkt des Schreibens dieser Arbeit liegt Cordova in der Version 4.1.2 vor. In dieser Version können Apps für Android 2.3.x und 4.x sowie für iOS >5.x erstellt werden \cite{cordova-android}\cite{cordova-ios}. 

\section{AngularJS}

AngularJS ist ein JavaScript-Framework, mit dem dynamische \gls{sp} Anwendungen erstellt werden können. Es wurde von Google veröffentlicht und erfreut sich derzeit stetig steigender Beliebtheit. Tatsächlich verfügt AngularJS über einige Features, die das Entwickeln von Webanwendungen sehr angenehm gestalten. Einzelne Seiten einer mit AngularJS erstellten Anwendung basieren auf der \gls{mvc}-Architektur. In JavaScript erstellte Controller steuern die Logik und vermitteln zwischen dem Model und der Benutzeroberfläche. Diese Benutzeroberfläche wird mit einem \gls{Markup}, das auf \gls{html} basiert, deklarativ beschrieben und präsentiert Inhalte aus dem Model. Durch das von AngularJS bereitgestellte Two-Way Databinding wird die Benutzeroberfläche automatisch mit dem Model synchronisiert. Änderungen im Model werden somit direkt in der Benutzeroberfläche sichtbar und Änderungen in der Benutzeroberfläche werden automatisch im Model hinterlegt. Zusätzlich zu den \gls{mvc} Komponenten ermöglicht AngularJS das Definieren von so genannten Services. Dies sind Objekte, die über die Laufzeit einer Anwendung bestimmte Funktionen übernehmen. Ein weiteres Feature von AngularJS sind Direktiven. Durch Direktiven lassen sich eigene \gls{html}-Elemente oder -Attribute definieren und in die Benutzeroberflächen integrieren. Dies fördert die Übersichtlichkeit und ermöglicht die Bildung von wiederverwendbaren Komponenten. AngularJS verwendet das \gls{di} Pattern, um Abhängigkeiten zwischen Komponenten einer Anwendung zu definieren und aufzulösen. Darüber lassen sich Services in beliebigen Komponenten der Anwendung einbinden und nutzen. In Kombination mit der \gls{mvc} Architektur ist es damit möglich, leicht testbare Apps zu entwickeln. Details zu diesen oder weiteren Features folgen in den übrigen Kapiteln dieser Arbeit.\cite{angularjs}

\section{Ionic}

Ionic ist ein Framework zur Entwicklung von grafischen Benutzeroberflächen für Apps auf Basis von Webtechnologien. Es stellt bekannte Komponenten für \glspl{ui} zur Verfügung, um die Entwicklung zu beschleunigen. Auf Wunsch sind diese Komponenten dem nativen \glqq Look and Feel\grqq{} der jeweiligen Plattform nachempfunden. Im Kern verwendet Ionic eine bestimmte Version von AngularJS und stellt über Direktiven die \gls{ui}-Komponenten zur Verfügung. Die aktuelle Version von Ionic heißt 1.0.0-rc.1. Dieser Name impliziert bereits, dass sich das Framework noch nicht in einer stabilen Version befindet. In dieser Version von Ionic wird iOS 6+, sowie Android 4.0+ offiziell unterstützt. Da Ionic auf AngularJS basiert, setzt jede Version von Ionic eine spezielle Version von AngularJS voraus. In diesem Fall ist es die Version 1.3.13.\cite{ionic}