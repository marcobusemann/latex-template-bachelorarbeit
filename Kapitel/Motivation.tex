\mbchapter{Motivation}
Das Interesse an Apps für Smartphones oder Tablets ist seit einigen Jahren ungebrochen. Und wohl auch in Zukunft ist mit zunehmendem Interesse zurechnen. Regelmäßig erscheinen neue Betriebssystemen oder Plattformen, die mit einer App abgedeckt werden müssen. Dadurch ergibt sich die Anforderung die jeweilige App auf möglichst vielen dieser Systeme zu positionieren. Unternehmen stehen bei der Entwicklung von neuen Apps vor der Frage, welche Technologien sie einsetzen um möglichst effektiv alle Plattformen abzudecken. In diesem Zusammenhang treten Webtechnologien immer mehr in den Fokus. Entwickler können meist auf jahrelange Erfahrung mit Webtechnologien zurückgreifen und im besten Fall alle Plattformen mit einer einzigen App abdecken. Doch trotzdem fällt oft eine Entscheidung gegen diese Technologien. Dies zeigt auch das Umdenken großer Unternehmen wie Facebook\cite{fb-enginering-backtonative} und LinkedIn\cite{li-engineering-backtonative} in diesem Zusammenhang. Die Performance, Usability und Entwicklertools reichen oft nicht an eine native Lösung heran \cite{mobilewebslow}. Moderne Frameworks auf Basis von Webtechnologien wie AngularJS und Ionic versuchen diese Lücke zu schließen und legen bei ihrer Umsetzung viel Wert auf eine performante und ressourcensparende Implementierung. Dadurch soll schon heute die Entwicklung konkurrenzfähiger Apps für Smartphone möglich sein. Um eine gewisse Performance zu gewährleisten reicht es jedoch nicht aus nur diese Frameworks einzusetzen. Die Performance hängt stark von der eigenen Implementierung ab. Aus diesem Grund werde ich in dieser Arbeit so genannte Best-Practice Lösungen für die Entwicklung von mobilen hybriden Web-Anwendungen mit AngularJS und Ionic erarbeiten. 
