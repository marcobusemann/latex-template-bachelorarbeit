\mbchapter{Plattformen}
\label{plattformen}
Die vorgestellten Frameworks setzen alle eine bestimmte Version von Android voraus. Im Folgenden wird auf die daraus resultierenden Probleme eingegangen.
Für Ionic ist eine Android Version größer 4.0 und bei Cordova entweder die Version 2.3.x oder ebenfalls eine Version größer 4.0 erforderlich. Daraus ergibt sich, dass mindestens Android 4.0 für den Einsatz dieser Frameworks benötigt wird.
\\\\
Wie bereits im vorherigen Kapitel beschrieben, verwendet Cordova eine \gls{WebView} Komponente, um die App darzustellen. Diese Komponente basierte seit dem Bestehen von Android auf der WebKit Engine \cite{android-webview-component}. Erst seit Android Version 4.4 wird die von Google entwickelte Chromium  Browser-Engine verwendet\cite{chromium}. Der Grund für diesen Wechsel war zum einen die bessere Performance und zum anderen eine bessere Unterstützung von Standards \cite{android-migrating-webview}\cite{android-webview}. Bei der Entwicklung von Apps mit Cordova wird demnach, je nach Android Version, eine unterschiedliche Browser-Engine für die Darstellung verwendet. Dadurch kann es zu Unterschieden in der Performance kommen. Außerdem muss sichergestellt sein, dass verwendete Standards in beiden Versionen funktionieren. 
\\\\
Die Tabelle \ref{android-versionsverteilung} zeigt die aktuelle Verteilung der eingesetzten Versionen von Android. Laut dieser Übersicht verwenden derzeit 52,7\%  aller Android-Geräte eine Version zwischen 4.0 und 4.3. Die neuen und für die hier betrachtete Art von Anwendung günstigeren Versionen ab 4.4, werden erst von 39,1\% aller Geräte eingesetzt \cite{android-plattform-verteilung}. Daraus lässt sich ableiten, dass die Zielgruppe einer App die erforderliche Performanceoptimierung maßgeblich beeinflusst.
\begin{table}
	\centering
	\begin{tabular}{l*{6}{c}r}
		\textbf{Version} & \textbf{Codename} & \textbf{API} & \textbf{Distribution} \\
		\hline
		2.2 & Froyo & 8 & 0.4\% \\
		2.3.3 - \\
		2.3.7 & Gingerbread & 10 & 6.9\% \\
		4.0.3 - \\
		4.0.4 & Ice Cream Sandwich & 15 & 5.9\% \\
		4.1.x & Jelly Bean & 16 & 17.3\% \\
		4.2.x &  & 17 & 19.4\% \\
		4.3 &  & 18 & 5.9\% \\
		4.4 & KitKat & 19 & 40.9\% \\
		5.0 & Lollipop & 21 & 3.3\%
	\end{tabular}
	\caption{Android Versionsverteilung
	\cite{android-plattform-verteilung}}
	\label{android-versionsverteilung}
\end{table}
Tim Roes, ein Computer Scientist, hat Benchmarks mit beiden Engines durchgeführt und die Ergebnisse auf seinem Blog veröffentlicht \cite{android-webview-benchmark}: 
\begin{quote}
"The new (chromium based) WebView is faster – so far no surprise. But looking at the numbers, the performance has really increased in several areas (like up to 354\% for HTML5 Canvas or 358\% for some Javascript test)."
\end{quote}
Er hat die gleichen Benchmarks auf demselben Gerät jeweils mit Android 4.3 und Android 4.4 durchgeführt. Abgesehen von Performanceoptimierungen, die durch den Versionssprung ausgelöst wurden, zeigt dieser Vergleich, dass vor allem für Versionen kleiner 4.4 von Android eine Optimierung relevant ist. 
\\\\
Verschiedene Frameworks versuchen diesen Vorteil auch für ältere Versionen von Android zu nutzen. Eines dieser Frameworks nennt sich  Crosswalk. Es ist ein vom Intel Open Source Technology Center veröffentlichtes Open Source Projekt. Es basiert auf Chromium und liefert für jede App eine eigene Browser-Engine mit. Eine mit Cordova entwickelte App verwendet dabei nicht die \gls{WebView} Komponente von Android, sondern eine von Crosswalk bereitgestellte Komponente. Somit basieren Apps ab Android 4.x auf der gleichen Browser-Engine und haben dadurch gleiche Voraussetzungen bezüglich der Performance und der verwendbaren Standards. Ein erster Schritt, um die Performance für ältere Versionen von Android zu verbessern, wäre demnach der Einsatz solch eines Frameworks. Es gibt jedoch auch Nachteile. Durch das Bereitstellen einer auf Chromium basierten Engine müssen einige Software-Komponenten, Bibliotheken und Binärdaten mitgeliefert werden \cite{crosswalk}. Dadurch beträgt die Größe der App ohne weitere Inhalte bereits \mytilde20mb. Es muss daher abhängig von der jeweiligen App entschieden werden, ob ein solches Verfahren sinnvoll ist.