\mbchapter{Einleitung}

\section{Motivation}
Das Interesse an Apps für Smartphones oder Tablets ist seit einigen Jahren ungebrochen. Auch in Zukunft ist mit zunehmendem Interesse zu rechnen. Regelmäßig erscheinen neue Betriebssysteme oder Plattformen, die mit einer App abgedeckt werden müssen. Dadurch ergibt sich die Anforderung die jeweilige App auf möglichst vielen dieser Systeme zu positionieren. Unternehmen stehen bei der Entwicklung von neuen Apps vor der Entscheidung, welche Technologien sie einsetzen, um möglichst effektiv alle Plattformen abzudecken. In diesem Zusammenhang treten Webtechnologien immer mehr in den Fokus. Entwickler können meist auf jahrelange Erfahrung mit Webtechnologien zurückgreifen und im besten Fall alle Plattformen mit einer einzigen App abdecken. Doch trotzdem fällt oft eine Entscheidung gegen diese Technologien. Dies zeigt das Umdenken großer Unternehmen wie Facebook\cite{fb-enginering-backtonative} und LinkedIn\cite{li-engineering-backtonative} in diesem Zusammenhang. Die Performance, Usability und Entwicklertools reichen oft nicht an eine native Lösung heran \cite{mobilewebslow}. Moderne Frameworks auf Basis von Webtechnologien wie AngularJS und Ionic versuchen, diese Lücke zu schließen und legen bei ihrer Umsetzung viel Wert auf eine performante und ressourcensparende Implementierung. Dadurch soll schon heute die Entwicklung konkurrenzfähiger Apps für Smartphones möglich sein. Um eine gewisse Performance zu gewährleisten, reicht eine einfache Verwendung dieser Frameworks jedoch meist nicht aus. Die Performance hängt stark von der eigenen Implementierung ab. Aus diesem Grund werde ich in dieser Arbeit so genannte Best-Practice Lösungen für die Entwicklung von mobilen hybriden Webanwendungen mit AngularJS und Ionic erarbeiten. 

\section{Zielsetzung}
Das Ziel dieser Arbeit ist die Ermittlung von Best-Practice Lösungen, um Apps für mobile Plattformen mit den Frameworks AngularJS und Ionic möglichst effizient zu implementieren. Der Fokus liegt dabei klar auf der Performance unterschiedlicher Implementierungsansätze. Das Ergebnis soll eine Erkenntnissammlung umfassen, die es Entwicklern ermöglicht, ohne weitere Recherchen eine gewisse Performance zu erreichen. 

\section{Aufbau der Arbeit}
Der Aufbau dieser Arbeit umfasst mehrere Schritte, um das angegebene Ziel zu erreichen. Zunächst werden die verwendeten Frameworks kurz beschrieben und auf konkrete Versionen eingeschränkt. Die Geschwindigkeit, mit der sich Webframeworks derzeit entwickeln, macht diese Konkretisierung erforderlich. Im Anschluss daran findet eine Betrachtung der Zielplattform statt. Im Speziellen fokussiert sich diese Arbeit dabei auf Android. Im nächsten Schritt wird der Begriff \glqq  Performance\grqq{} definiert und untersucht, welche Kategorien von Performance Problemen es allgemein und mit Fokus auf Webanwendungen gibt. Um einen theoretischen Background aufzubauen, findet dabei ebenfalls eine Betrachtung der Funktionsweisen aktueller Browser-Engines statt. Es werden die Ziele hinsichtlich der Performance definiert und anschließend die Durchführung von Messungen im Rahmen dieser Arbeit beschrieben. Für die Tests der einzelnen Untersuchungen ist es notwendig, eine Beispielapplikation zu erstellen, worin die Best-Practice Lösungen implementiert und deren Zeiten gemessen werden. Das darauf folgende Kapitel befasst sich daher mit dem Aufbau dieser Anwendung. Im Hauptteil werden die unterschiedlichen Ansätze beschrieben, verglichen und in einem Ergebnis zusammengefasst. 