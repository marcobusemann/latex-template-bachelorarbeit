% toc=listofnumbered, sorgt dafür, dass unter anderem die Abbildungsverzeichnisse im 
%                     Inhaltsverzeichnis mit aufgenommen wird.
\documentclass[bibtotoc,toc=listofnumbered,BCOR12mm,DIV11,titlepage,a4paper,oneside]{scrbook}

% Zeilenabstand
\usepackage{setspace}

%
\usepackage{etoolbox}

% Language packages
\usepackage[T1]{fontenc}
\usepackage[utf8]{inputenc}
\usepackage[ngerman]{babel}
\usepackage[babel, german=quotes]{csquotes}

% Bilder einbinden
\usepackage{graphicx}

% Abkürzungsverzeichnis
\usepackage[printonlyused, footnote]{acronym}

% Literaturverzeichnis
\usepackage{biblatex}
\bibliography{Sonstiges/Literatur}

% PDF-Informationen und Hyperlinks
\usepackage[pdftex,plainpages=false,pdfpagelabels,
            pdftitle={Bachelorarbeit},
            pdfauthor={Vorname Name}
            ]{hyperref}

% Glossar
\usepackage[section,nopostdot,acronym]{glossaries}

% Beschreibung beim ersten auftauchen in die Fusszeile schreiben
\defglsdisplayfirst[main]{\Glsentryname{\glslabel}\footnote{\Glsentrydesc{\glslabel}}}
\defglsdisplayfirst[acronym]{
	\Glsentryfirst{\glslabel}
	\ifglshaslong{\glslabel g} % stellt sicher, dass acronyme ohne glossar keine fußzeile haben.
		{}
		{\footnote{\Glsentrydesc{\glslabel g}}}}
\makeglossaries

% Generell information
\author{Vorname Nachname}
\title{Bachelorarbeit}

% Header konfigurieren
\usepackage{fancyhdr}
\pagestyle{fancy} 
\renewcommand*{\headrulewidth}{0.4pt} 
\lhead{} %Kopfzeile links
\chead{} %Kopfzeile mitte
\rhead{\thepage} %Kopfzeile rechts
\lfoot{} %Fusszeile links
\cfoot{} %Fusszeile mitte
\rfoot{} %Fusszeile rechts
\renewcommand*{\indexpagestyle}{fancy}
\renewcommand*{\partpagestyle}{empty}
\renewcommand*{\chapterpagestyle}{fancy}

% Sonstiges
\usepackage{array}

% JavaScript listings
%\usepackage{xcolor}
\usepackage[table,xcdraw]{xcolor}
\usepackage{textcomp}
\usepackage{caption}
\usepackage{listings}

% Überschriften für Listings
%\DeclareCaptionFont{white}{ \color{white} }
%\DeclareCaptionFormat{listing}{
%	\colorbox[cmyk]{0.43, 0.35, 0.35,0.01 }{
%		\parbox{\textwidth}{\hspace{15pt}#1#2#3}
%	}
%}
%\captionsetup[lstlisting]{ format=listing, labelfont=white, textfont=white, singlelinecheck=false, margin=0pt, font={bf,footnotesize} }
%\geometry{left=1.0in,right=1.0in,top=1.0in,bottom=1.0in }

% JavaScript Listing definieren
\lstdefinelanguage{JavaScript} {
	morekeywords={
		break,const,continue,delete,do,while,export,for,in,function,
		if,else,import,in,instanceOf,label,let,new,return,switch,this,
		throw,try,catch,typeof,var,void,with,yield
	},
	sensitive=false,
	morecomment=[l]{//},
	morecomment=[s]{/*}{*/},
	morestring=[b]",
	morestring=[d]'
}

% Formatieren
\definecolor{mygreen}{rgb}{0,0.6,0}
\definecolor{mygray}{rgb}{0.5,0.5,0.5}
\definecolor{mymauve}{rgb}{0.58,0,0.82}
\lstset{
	frame=top,frame=bottom,
	basicstyle=\footnotesize\ttfamily,    % the size of the fonts that are used for the code
	stepnumber=1,                           % the step between two line-numbers. If it is 1 each line will be numbered
	numbersep=10pt,                         % how far the line-numbers are from the code
	tabsize=2,                              % tab size in blank spaces
	extendedchars=true,                     %
	breaklines=true,                        % sets automatic line breaking
	captionpos=t,                           % sets the caption-position to top
	mathescape=true,
	stringstyle=\color{white}\ttfamily, % Farbe der String
	showspaces=false,           % Leerzeichen anzeigen ?
	showtabs=false,             % Tabs anzeigen ?
	xleftmargin=17pt,
	framexleftmargin=17pt,
	framexrightmargin=0pt,
	framexbottommargin=5pt,
	framextopmargin=5pt,
	showstringspaces=false      % Leerzeichen in Strings anzeigen ?
	backgroundcolor=\color{white},
	commentstyle=\color{mygreen},    % comment style
	keywordstyle=\color{blue},       % keyword style
	language=JavaScript,                 % the language of the code
	rulecolor=\color{black},
	stringstyle=\color{mymauve},     % string literal style
	numbers=left,                    % where to put the line-numbers; possible values are (none, left, right)
	numbersep=5pt,                   % how far the line-numbers are from the code
	numberstyle=\tiny\color{mygray}, % the style that is used for the line-numbers
	mathescape=false
}

% Zusätzliche Schlüsselworte für HTML-Listing definieren. 
% Bei der Verwendung mit der style eingebunden werden.
\lstdefinestyle{ionicHtmlStyle}{
	alsoletter={-},
	morekeywords={
		ng-repeat,
		ion-list,
		ion-item,
		ng-include,
		ng-src,
		ng-class,
		bo-src,
		bo-class,
		bo-text,
		ion-infinite-scroll,
		on-infinite,
		distance,
		ng-if,
		placeholder,
		ng-model
	} 
}

\DeclareCaptionFormat{listing}{\rule{\dimexpr\textwidth+0pt\relax}{0.4pt}\par\vskip0pt#1#2#3}
\captionsetup[lstlisting]{format=listing,singlelinecheck=false, margin=0pt, font={sf},labelsep=space,labelfont=bf}

% Listings unterstützen kein UTF-8 encoding. Daher wird hiermit gesagt, 
% wie die meist verwendeten Zeichen interpretiert werden sollen.
\lstset{literate=
	{á}{{\'a}}1 {é}{{\'e}}1 {í}{{\'i}}1 {ó}{{\'o}}1 {ú}{{\'u}}1
	{Á}{{\'A}}1 {É}{{\'E}}1 {Í}{{\'I}}1 {Ó}{{\'O}}1 {Ú}{{\'U}}1
	{à}{{\`a}}1 {è}{{\`e}}1 {ì}{{\`i}}1 {ò}{{\`o}}1 {ù}{{\`u}}1
	{À}{{\`A}}1 {È}{{\'E}}1 {Ì}{{\`I}}1 {Ò}{{\`O}}1 {Ù}{{\`U}}1
	{ä}{{\"a}}1 {ë}{{\"e}}1 {ï}{{\"i}}1 {ö}{{\"o}}1 {ü}{{\"u}}1
	{Ä}{{\"A}}1 {Ë}{{\"E}}1 {Ï}{{\"I}}1 {Ö}{{\"O}}1 {Ü}{{\"U}}1
	{â}{{\^a}}1 {ê}{{\^e}}1 {î}{{\^i}}1 {ô}{{\^o}}1 {û}{{\^u}}1
	{Â}{{\^A}}1 {Ê}{{\^E}}1 {Î}{{\^I}}1 {Ô}{{\^O}}1 {Û}{{\^U}}1
	{œ}{{\oe}}1 {Œ}{{\OE}}1 {æ}{{\ae}}1 {Æ}{{\AE}}1 {ß}{{\ss}}1
	{ç}{{\c c}}1 {Ç}{{\c C}}1 {ø}{{\o}}1 {å}{{\r a}}1 {Å}{{\r A}}1
	{€}{{\EUR}}1 {£}{{\pounds}}1
}

% Tabellen mit merged rows
\usepackage{multirow}

% TODO management
\newcommand{\todo}[1]{\textcolor{red}{@TODO: #1}}

% Tilde
\newcommand{\mytilde}{{\raise.17ex\hbox{$\scriptstyle\mathtt{\sim}$}}}

% Nummerierung mit \subsubsection
\setcounter{secnumdepth}{3}
\setcounter{tocdepth}{3} 

%
\usepackage{geometry}
\geometry{left=40mm,top=30mm,right=30mm,bottom=30mm}